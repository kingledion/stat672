\documentclass{article}
\usepackage{amsmath, amsfonts, amssymb}
\usepackage[margin=0.5in]{geometry}
\usepackage{graphicx}
\graphicspath{ {images/} }

\setlength{\parskip}{\baselineskip}%
\setlength{\parindent}{0pt}%

\title{Homework 6}
\author{Daniel Hartig}


\begin{document}
\maketitle

\subsection*{1 a.}

Each decision stump classifies by assigning all values either greater than or less than $t$ to one label class, and the rest of the samples to the other label class. The decision stump must both optimize the value of $t$ and the greater than or less than comparator, which is the $c_1$ and $c_2$ component. This stump will only use one feature to make its decision.

A stump has very little power for good classification. Firstly, it is only using a single feature. If the real relationship between the features and predicted outcomes is complex, any method using a single feature would be barely better than random guessing. Secondly, the stump classifies based solely on the comparison relationship to $t$, so its power is not great. 





\end{document}
