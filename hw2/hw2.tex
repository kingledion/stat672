\documentclass{article}
\usepackage{amsmath}
\usepackage[margin=0.5in]{geometry}

\setlength{\parskip}{\baselineskip}%
\setlength{\parindent}{0pt}%

\title{Homework 2}
\author{Daniel Hartig}


\begin{document}
\maketitle

\section*{Problem 1a}

The Bayes classification function ($f^*$) for a piecewise function is the most likely outcome for each case. Thus for
$$ P(Y = 1|X = x) = \begin{cases} 
0.9,& x < 0.2, \\
0.2,& 0.2 < x < 0.8,\\
0.9,& x > 0.8,\\ \end{cases} $$ then
$$ f^* = \begin{cases} 
1,& x < 0.2, \\
-1,& 0.2 < x < 0.8,\\
1,& x > 0.8.\\ \end{cases} $$ The risk of the Bayes classification function is the probability the classification function not equalling the actual distribution of Y, or
$$ R(f^*) = \begin{cases} 
0.1,& x < 0.2, \\
0.2,& 0.2 < x < 0.8,\\
0.1,& x > 0.8.\\ \end{cases} $$
Integrating the Bayes risk function over the space ${0, 1}$ gives $0.1\cdot0.2 + 0.2\cdot0.6+0.1\cdot0.2 = 0.16$.

\section*{Problem 1b}

In order for the excess risk to be zero, the classifier must have the same sign as the Bayes classification function ($f^*$) in the entire domain of $\mathcal{X}$, which is $[0, 1]$. For the specific $f^*$ given in Problem 1a, there are two changes of sign between 0 and 1. Since our function space ($\mathcal{F}$) is all polynomial functions of degree $d$, then this polynomial must have exactly two roots in the range $[0, 1]$.

For any polynomial with degree $d < 2$, there cannot be 2 roots, the classifier cannot have the same sign as $f^*$ on the domain of $\mathcal{X}$, and thus the excess risk cannot be zero.

For the set of all polynomial with degree $d >2$ there exists at least at least one polynomial with exactly two roots in the range $[0, 1]$. For $d=2$ and $f^*$ given above, any polynomial with roots $0.2$ and $0.8$ and no other roots in $[0, 1]$ will have excess risk equal to zero.
\end{document}